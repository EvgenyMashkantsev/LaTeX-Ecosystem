% !TeX encoding = UTF-8
\documentclass[a4paper,12pt]{report}

\usepackage{cmap}
\usepackage[T1,T2A]{fontenc}
\usepackage[utf8]{inputenc}
\usepackage[english,russian]{babel}
\usepackage{indentfirst}
\usepackage{bm}
\usepackage[rgb]{xcolor}
\usepackage{listings}
\usepackage{float}
\usepackage{lscape}

% Математика
\usepackage{amsmath,amsfonts,bbm,gensymb,amssymb,amsthm,commath,mathtools} 

% Изображения
\usepackage{graphicx}
% \usepackage{wrapfig}
\usepackage[export]{adjustbox}

\definecolor{mygreen}{rgb}{0,0.6,0}
\definecolor{mygray}{rgb}{0.5,0.5,0.5}
\definecolor{mymauve}{rgb}{0.58,0,0.82}

\usepackage{hyperref}
\hypersetup{				% Гиперссылки
	colorlinks=true,       	% false: ссылки в рамках
	urlcolor=black          % на URL
}
% \usepackage[style=unsrt]{biblatex}
% \addbibresource{isostd.bib}

\title{Введите название Вашего труда}
\author{Как Вас зовут?}

\begin{document}
\maketitle
\begin{abstract}
Данный труд посвящён вопросам, которые относятся к очень интересным вопросам, но от которых могут возникнуть другие не менее интересные вопросы. Больше вопросов -- больше ответов, а это хорошо, не так ли? Кстати, то же вопрос.
\end{abstract}
\tableofcontents
\setlength{\parindent}{1.5em}
\chapter{Вступление}

Вначале было слово, и слово было Кнут. Дональд Кнут. Он выпустил систему компьютерной вёрстки \TeX в 1978 году. Молодец ли он? Конечно, молодец! А в 1984 году Лэсли Лэмпортом был разработан пакет макрорасширений для \TeX под названием \LaTeX. Разве не полезно? Полезнее, чем сырокопчёная колбаса!

\chapter{Заключение}
Мы разобрали больше чем все вопросы, и мы молодцы. Осталось разобрать ответы.

\bibliographystyle{unsrt}
\bibliography{isostd, global-security, global-security-systems-engineering, existential-risk-bibliography, other} 

\end{document}
